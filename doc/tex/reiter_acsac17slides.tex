\documentclass{beamer}
\usepackage{algorithm2e}
\usepackage{listings}
\usepackage{hyperref}
\usepackage{amsthm}
\newtheorem{conj}{Conjecture}[section]
\let\footnotesize\tiny

%\title{Pruneskin: Fuzzing Program Slices}
%\author{Andrew R. Reiter\\
%\tiny CA Veracode}
%\date{\today}

\begin{document}

%\frame{\titlepage}

%\section{Investigating }
\frame
{
	\frametitle{Pruneskin: Investigation into Fuzzing Program Slices}
	\tiny
	
	\fboxsep=0pt
		\noindent\hbox{%
	\begin{minipage}[t]{0.48\linewidth}
	High-level version of algorithm:
	\begin{algorithm}[H]
		\SetAlgoLined
		\KwData{Program P, sample input I}
		\KwResult{$C = $\{$c \in inputs(P) \vert exec(P,c) \rightarrow$ crash\}}
		$A \leftarrow Analysis(P, I)$\;
		$S_d \leftarrow Slice(P, A, $ depth\footnote{The reasoning for \emph{depth} is that the goal is to fuzz a subgraph of a graph and not separate.} $d)$\;
		$\{c'_{d,i}\}_i \leftarrow Fuzz(S_d, I)$\;
		\For{$c' \in \{c'_{d,i}\}_i$}{
			$c \leftarrow MapCrash(P, S_d, c')$\;
			$C.append(c)$\;
		}
	\end{algorithm}\end{minipage}}%
\hfill%
\fbox{
	\begin{minipage}[t]{0.48\linewidth}	
	\begin{conj} There exists a \text{Slice} algorithm such that the \text{MapCrash} algorithm either
	\begin{itemize}
	\item acts like Identity on input sample, or
	\item maps samples crashing $S_d$ to samples crashing $P$ with minimal difficulty or human interaction (heh).
	\end{itemize}
	\end{conj}
	\end{minipage}
}

\vfill	
	A  \textbf{naive} starting approach is implemented:
	\begin{itemize}
	\item For \emph{Analysis()}: execution flow from a given desirable input (i.e. execution trace log)
	\item For \emph{Slice()}: attempt to remove the non-exec'd basic blocks of functions reached, unless they are the destination
	block for a compare-branch that we wish to keep (such as if (read() < 0) ...).
	\item For \emph{MapCrash()}: fuzz original $P$ with input samples that are crashes from fuzzing $S_d$. 
	with the idea being that under some measurable fuzzing space $\left\vert d(c) - d(c') \right\vert \leq$ something reasonable.
	\end{itemize}
	and using afl-fuzz.  But, many possibilities for Analysis, Slice, Fuzz, \& MapCrash functions. 
	
	
	Find this code and horseplay: \url{https://github.com/roachspray/acsac17wip/blob/master/naive}
}

\frame
{
	\frametitle{Minimal set of results}
	\tiny

	
	\begin{center}
	\begin{tabular}{| c | c | c  | c | c | c |}
	\hline
	Program            & Input\footnote{all in repo}        & Slice Depth  & P 5h fuzz & S\_d 1h fuzz & Mapped 4h fuzz\footnote{uses inputs from slice fuzz output} \\ \hline
	cracklib-2.9.2-5 & smalldict   &  1                 & 0             & 3                    & 0  \\
	                 	&                  &  2 		  & 0 		   & 0 			& 0  \\
	                 	&             	   & 	3 		 & 0 		   & 0 			& 0  \\ \hline
	exifprobe current & tryone.jpg & 3 		 & 4 		   & 27 			& 37  \\ 
	                  &            & 4 & 4 & 30 & 50  \\ 
	                  &            & 5 & 4 & 27 & 13  \\ 
	                  &            & 6 & 4 & 28 & 30  \\ \hline
	tcpdump current   & ntp.pcap   & 6 & 0 & 0 & 0   \\ 
	                  &   & 7 & 0 & 0 & 0   \\ 
	                  &    & 8 & 0 & 0 & 0   \\ 
	                  &    & 9 & 0 & 0 & 0   \\ \cline{2-6}
                      & hcrt.pcap  & 7 & 0 & 1 & 0   \\
                      &            & 8 & 0 & 6 & 0   \\ 
                      &            & 9 & 0 & 0 & 0   \\
                      &            & 10 & 0 & 0 & 0  \\ \cline{2-6}
                      & arp.pcap   & 3 & 0 & 0 & 0  \\ 
                      &    & 4 & 0 & 0 & 0  \\                    
                      &    & 5 & 0 & 0 & 0  \\ 
                      &    & 6 & 0 & 0 & 0  \\
                      &    & 7 & 0 & 0 & 0  \\
                      &    & 8 & 0 & 0 & 0  \\ \hline                                                                 
	\end{tabular}
	\end{center}


	Fundamental issues with the above:	
	\begin{itemize}
	\item Timebox time selections and averaging of timebox results (handle randomness)
	\item How to deal with (not) having groundtruth? Equality of bugs found, etc pose non-trivial issues. I yen for a
	dataset of code, crash, time-til-found, hardware, etc etc.
	\item Not listed: slice sizes seem invalid
	\end{itemize}
	
}



\frame
{
	\frametitle{Next Steps}
	\tiny
	\begin{itemize}
	\item Improve \emph{Analysis} step. Initially included static value flow analysis, but the code was not working. Investigate clang analyze or hand developed. Also include the use of either manticore\footnote{\url{https://github.com/trailofbits/manticore}} or libdft\footnote{\url{https://www.cs.columbia.edu/~vpk/research/libdft/}}.


	\item Improve \emph{Slice} step to use data flow information in block / function removal selection
	\item Improve \emph{Slice} step by better dead code removal

	\item Investigate  possible \emph{MapCrash} methods:
	\begin{itemize}
	\tiny
	\item Cascade AFL 
\[
(C_d \leftarrow Fuzz(S_d, In)) \rightarrow (C_{d-1} \leftarrow Fuzz(S_{d-1}, C_d)) \rightarrow \ldots \rightarrow (C_0 \leftarrow Fuzz(P, C_1))
\]
	
	\item Selective AFL instrumentation based on slicing information. See \footnote{https://github.com/roachspray/acsac17wip/blob/master/patches/afl-2.51b-optionally-instrument-based-on-trace.patch}
	\item Investigate signaling AFL from undesirable paths. Equivalent to VUZZer with  weighting.
	\item Generate crash trace from sample, use with manticore to guide program through solving CSPs as it goes toward crash state.
	\end{itemize}
	\end{itemize}
	As you can tell, this work is still in it's infancy and there are many paths to investigate. Trying to determine best routes and
	choices to keep the science legit is an important piece (i.e. some understanding of a ground truth). But, many other areas to play.
	I would greatly appreciate anyone interested to combine forces, or outright take some ideas and go forth.
	
}
\end{document}
